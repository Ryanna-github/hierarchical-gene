% XeLaTeX can use any Mac OS X font. See the setromanfont command below.
% Input to XeLaTeX is full Unicode, so Unicode characters can be typed directly into the source.

% The next lines tell TeXShop to typeset with xelatex, and to open and save the source with Unicode encoding.

%!TEX TS-program = xelatex
%!TEX encoding = UTF-8 Unicode

\documentclass[a4paper,master]{ructhesis}

%自添宏包
\usepackage{skak}%国际象棋
\usepackage{subfigure}%子图http://www.ctex.org/documents/latex/graphics/node111.html
%\usepackage{chemfig}%化学式
\usepackage{ctex}
\usetikzlibrary{trees}
\usepackage{amsmath}
\usepackage[ruled, lined, linesnumbered, commentsnumbered, longend]{algorithm2e}
\usepackage{tabularx}
\usepackage{setspace}
\newtheorem{definition}{Definition}



%将封面信息补全,相关专业名称过长的请在文字前添加命令\ziju{-0.15}
%文头
%\sign{中国人民大学本科毕业论文}
\sign{硕士学位论文}
%\sign{博士学位论文}



%以下信息本科研究生都需要补全
\title{带有嵌套结构的高维数据有限混合回归模型研究}%论文题名
\author{任焱}%作者
\school{统计学院}%学院
\field{统计学}%专业
\studentid{2021103739}%学号
\advisor{孙怡帆}%指导老师
\date{2024年3月21日}


%以下本科填写
\grade{}%年级
\score{}%成绩
\thesiscode{论文编码:RUC-BK-专业代码}%论文编码
\subtitle{}%论文副题名,没有不填写


%以下研究生填写
\etitle{Hierarchical Finite Mixture Regression Analysis with High Dimensional Data}%英文题目
\keywords{有限混合模型;亚组识别;高维数据}%论文主题词
%摘要关键词
\keywordzh{有限混合模型;亚组识别;高维数据}%中文摘要关键词
\keyworden{Finite Mixture Regression\qquad Subgroup Identification\qquad High dimensional data}%英文摘要关键词


%
\begin{document}

%扉页
\maketitle

%独创性声明
\originality
%授权书在这插入
%\authorization{figures/shouquan.png}
%中文摘要
\include{format/cabstractpage}
%英文摘要
\include{format/eabstractpage}


\frontmatter

%正文目录
\tableofcontents
%插图目录
\listoffigures
%表格目录
\listoftables


\mainmatter\clearpage
\pagestyle{fancy}

%正文章节
\include{chap/chapter1}
\include{chap/chapter2}
\include{chap/chapter3}
\include{chap/chapter4}
\include{chap/chapter5}
\include{chap/chapter6}
%\input{chap/chapter4}%要插入本科签名的最后一个章节,插入命令使用\input{}

%本科签名
%\autograph


%参考文献
\bibliographystyle{ref/rucbib}
\setcitestyle{super,square,comma,sort&compress}
\bibliography{ref/test}
\nocite{*}
\addcontentsline{toc}{chapter}{参考文献}

%附录
%\appendix
%\include{chap/appendix_1}

%致谢
\include{format/acknowledge}



\end{document}  






